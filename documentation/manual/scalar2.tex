\lineiii{APPNAME}
{string}
{name of executable as given in the \member{APP\_NAME} attribute in the OS object} 
\lineiii{ARCH}
{string}
{name of the architecture. This is the first item in the target.} 
\lineiii{ASSEMBLEREXE}
{string}
{name of the assembler executable used. This is the \member{ASSEMBLER} attribute in the OS object. It is set to {\em as} by default. It is used for build dependent templates.} 
\lineiii{ASSEMBLER}
{string}
{name of the assembler used. This is the \member{ASSEMBLER} attribute in the \member{MEMMAP} attribute of the OS object. It is used for assembler dependent templates.} 
\lineiii{AUTOSAR}
{boolean}
{true if Trampoline is compiled with the Autosar extension.} 
\lineiii{BOARD}
{string}
{name of the board. This is the third item (if any) in the target.} 
\lineiii{CHIP}
{string}
{name of the chip. This is the second item (if any) in the target.} 
\lineiii{COMPILEREXE}
{string}
{name of the compiler executable used. This is the \member{COMPILER} attribute in the OS object. It is set to {\em gcc} by default. It is used for build dependent templates. Do not confuse with the \member{COMPILER} data.} 
\lineiii{COMPILER}
{string}
{name of the compiler used. This is the \member{COMPILER} attribute in the \member{MEMMAP} attribute of the OS object. It is used for compiler dependent templates.} 
\lineiii{CPUNAME}
{string}
{name given to the OIL CPU object} 
\lineiii{EXTENDED}
{boolean}
{true if Trampoline is compiled in extended error handling mode.} 
\lineiii{FILENAME}
{string}
{the name of the file which will be written as the result of the computation of the current template.} 
\lineiii{FILEPATH}
{string}
{the full absolute path of the file which will be written as the result of the computation of the current template.} 
\lineiii{NATIVEFILEPATH}
{string}
{the full absolute path of the file which will be written as the result of the computation of the current template in native OS format.} 
\lineiii{ITSOURCESLENGTH}
{integer}
{number of interrupt sources as defined in the \file{target.cfg} file.} 
\lineiii{LINKEREXE}
{string}
{name of the linker executable used. This is the \member{LINKER} attribute in the OS object. It is set to {\em gcc} by default. It is used for build dependent templates. Do not confuse with the \member{LINKER} data.} 
\lineiii{LINKER}
{string}
{name of the linker used. This is the \member{LINKER} attribute in the \member{MEMMAP} attribute of the OS object. It is used for linker dependent templates.} 
\lineiii{LINKSCRIPT}
{string}
{name of the link script file as given in the \member{MEMMAP} attribute of the OS object.} 
\lineiii{MAXTASKPRIORITY}
{integer}
{the highest computed priority among the tasks.} 
\lineiii{OILFILENAME}
{string}
{name of the root OIL source file} 
\lineiii{PROJECT}
{string}
{name of the project. The name of the project is the \programopt{p} (or \longprogramopt{project}) value if it is set or the name of the oil file without the extension.} 
\lineiii{SCALABILITYCLASS}
{integer}
{the Autosar scalability class used by the application. If Autosar is not enabled, \member{SCALABILITYCLASS} is set to 0.} 
\lineiii{TARGET}
{string}
{name of the target. This is the \programopt{t} (or \longprogramopt{target}) option value of goil.} 
\lineiii{TEMPLATEPATH}
{string}
{path to the template root directory. This is the \longprogramopt{templates} option value of goil or the value of the \envvar{GOIL\_TEMPLATES} environment variable.} 
\lineiii{TIMESTAMP}
{string}
{current date} 
\lineiii{TRAMPOLINEPATH}
{string}
{path to the trampoline root directory. This is the \member{TRAMPOLINE\_BASE\_PATH} attribute of the OS object. It defaults to ``..".} 
\lineiii{USECOMPILERSETTINGS}
{boolean}
{true if memory mapping is enabled (Goil generates the \file{Compiler.h} and \file{Compiler_Cfg.h} files and Trampoline includes them).} 
\lineiii{USEBUILDFILE}
{boolean}
{true if a build file is used for the project ie option \programopt{g} or \longprogramopt{generate-makefile} is given.} 
\lineiii{USECOM}
{boolean}
{true if the application uses OSEK COM.} 
\lineiii{USEERRORHOOK}
{boolean}
{true if Trampoline uses the Error Hook.} 
\lineiii{USEGETSERVICEID}
{boolean}
{true if Trampoline uses the service ids access macros.} 
\lineiii{USEINTERRUPTTABLE}
{boolean}
{true if the wrapping of interrupt vector to glue functions used to increment a counter or to activate an ISR2 (for instance) should be generated. The actual code generation is up to the port.} 
\lineiii{USELOGFILE}
{boolean}
{true if goil generates a log file, ie option \programopt{l} or \longprogramopt{logfile} is given.}
\lineiii{USEMEMORYMAPPING}
{boolean}
{true if memory mapping is enabled (Goil generates the \file{MemMap.h} file and Trampoline includes it).} 
\lineiii{USEMEMORYPROTECTION}
{boolean}
{true if Trampoline uses the Memory Protection.} 
\lineiii{USEOSAPPLICATION}
{boolean}
{true if Trampoline uses OS Applications.} 
\lineiii{USEPARAMETERACCESS}
{boolean}
{true if Trampoline uses the parmaters access macros.} 
\lineiii{USEPOSTTASKHOOK}
{boolean}
{true if Trampoline uses the Post-Task Hook.} 
\lineiii{USEPRETASKHOOK}
{boolean}
{true if Trampoline uses the Pre-Task Hook.} 
\lineiii{USEPROTECTIONHOOK}
{boolean}
{true if Trampoline uses the Protection Hook.} 
\lineiii{USERESSCHEDULER}
{boolean}
{true if Trampoline uses the RES_SCHEDULER resource.} 
\lineiii{USESHUTDOWNHOOK}
{boolean}
{true if Trampoline uses the Shutdown Hook.} 
\lineiii{USESTACKMONITORING}
{boolean}
{true if Trampoline uses the Stack Monitoring.} 
\lineiii{USESTARTUPHOOK}
{boolean}
{true if Trampoline uses the Startup Hook.} 
\lineiii{USESYSTEMCALL}
{boolean}
{true if services are called using a System Call (i.e. a software interrupt).} 
\lineiii{USETIMINGPROTECTION}
{boolean}
{true if Trampoline uses Timing Protection.} 
\lineiii{USETRACE}
{boolean}
{true if tracing is enabled.} 
